\documentclass[a4paper,twocolumn]{article}
\usepackage{hyperref}
\usepackage{amsmath}
\usepackage{amsfonts}
\usepackage{amssymb}
\usepackage[numbers,sort&compress]{natbib}

\title{Cosmology -- Theory}

\author{Sandro Dias Pinto Vitenti and Mariana Penna-Lima}

\date{\today}

\newcommand{\He}{\text{He}}
\newcommand{\Hen}{\He_\text{n}}
\newcommand{\Hy}{\text{H}}
\newcommand{\Hyn}{\Hy_\text{n}}
\newcommand{\e}{\text{e}}

\begin{document}

\maketitle

\section{Recombination}

In this section define our notation with respect to the recombination module, more details in~\cite{Weinberg2008}. We refer to the total number of hydrogen nucleus (ionized or not) as 
\begin{align}
n_{\Hyn} \equiv n_p,
\end{align}
where $n_p$ is the number or protons. We refer to the hydrogen atoms as $n_{\Hy}$ and ionized hydrogen as $n_{\Hy^+}$ and therefore $n_{\Hy} + n_{\Hy^+} = n_{\Hyn}$. In the same way for the helium the number of helium nuclei is $n_{\Hen}$ and the single and double ionized as $n_{\He^+}$ and $n_{\He^{++}}$ respectively.

We also we the helium primordial abundance as the ratio of the helium mass to the total baryonic mass, i.e.,
\hypertarget{def_Y_p}{}
\begin{align}
Y_p = \frac{n_{\Hen} m_{\He}}{(n_{\Hen} m_{\He} + n_{\Hyn} m_{\Hy})},
\end{align}
where $m_\Hy$ and $m_{\He}$ are the hydrogen and helium mass.

The element abundances are define as the ratio of the element by the total number of protons:
\begin{align}
X_{f} = \frac{n_{f}}{n_p}, 
\end{align}
where $f = (\e, \Hy,\;\Hy^+,\;\He,\;\He^+,\;\He^{++})$ and $(\e)$ represent the free electrons. These fractions have the following properties:
\begin{align}
X_\Hy + X_{\Hy^+} &= 1, \\
X_{\He} + X_{\He^+} + X_{\He^{++}} &= X_{\Hen}, \\
X_{\Hen} &\equiv \frac{m_p}{m_{\He}}\frac{Y_p}{1-Y_p}.
\end{align}
We also define the number of free electrons as $n_\e$. Assuming a neutral universe we have 
\begin{align}
X_\e = X_{\Hy^+} + X_{\He^{+}} + 2X_{\He^{++}}.
\end{align}

\section{Large Scale Structure}

\subsection{Window Function}
\hypertarget{sec_wf}{}
In order to study the statistical properties of the density fluctuation field at a certain 
scale $R$, we use the window function. As an example, to compute the variance of the density 
contrast at scale R, we convolve the window function in the Fourier space with the power spectrum.

\subsubsection{Top Hat}
\hypertarget{sec_wf_th}{}
This function returns the top hat window function in the real space.
\hypertarget{eq_th_real}{}
\begin{eqnarray}
  W_{TH}(r, R) = \frac{3}{4\pi R^3} \left\{ \begin{array} {ll}
         1 & \mbox{$\leq$ R}\\
        0 & \mbox{$>$ R.}
         \end{array} \right.
\end{eqnarray}
The mass enclosed within the volume selected by this window function is $M_{TH}(R)= 
\frac{4\pi}{3}\overline{\rho} R^3$,where $\overline{\rho}(z)$ is the mean density of the 
universe at redshift $z$.

The top-hat window function in the Fourier space is given by
\hypertarget{eq_th_fourier}{}
\begin{eqnarray}
W_{th}(k, R) &=& \frac{3}{(kR)^3}(\sin kR - (kR)\cos kR) \\
            &=& \frac{3}{(kR)} j_1(kR),
\end{eqnarray}
where $j_\nu(kR)$ is the spherical Bessel function. 

The first derivative with respect to $R$
\hypertarget{eq_th_fourier_der}{}
\begin{equation}
\frac{dW_{TH}(k, R)}{dR} = \frac{-9}{k^3 R^4} (\sin kR - (kR)\cos kR) + \frac{3}{k R^2} \sin kR.
\end{equation}

\subsubsection{Gaussian}
\hypertarget{sec_wf_gauss}{}
This function returns the gaussian window function in the real space,
\hypertarget{eq_gauss_real}{}
\begin{equation}
W_G(r, R) = (2 \pi R^2)^{-3/2}\exp \left( \frac{-r^2}{2 R^2} \right).
\end{equation}
The mass enclosed within the volume selected by this window function is 
 $ M_G(R) = (2\pi)^{3/2}\overline{\rho}(z) R^3$, where $\overline{\rho}(z)$
 is the mean density of the universe at redshift $z$.
 
This function returns the gaussian window function in the Fourier space,
\hypertarget{eq_gauss_fourier}{}
\begin{equation}
W_G(k, R) = \exp \left( \frac{-k^2 R^2}{2} \right).
\end{equation} 

This function returns the derivative with respect to R of the gaussian window function 
in the real space,
\hypertarget{eq_gauss_fourier_der}{}
\begin{equation}
\frac{dW_G(k, R)}{dR} = -k^2 R \exp \left( \frac{-k^2 R^2}{2} \right).  
\end{equation}

\subsection{Transfer Function}
\hypertarget{sec_transf}{}

\subsubsection{NcTransferFuncBBKS}

\subsubsection{NcTransferFuncEH}

\subsubsection{NcTransferFuncCAMB}

\subsubsection{NcTransferFuncPert}

\bibliography{references}
\bibliographystyle{apsrev}

\end{document}
